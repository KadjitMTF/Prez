\section{Язык Go}
\subsection{С чего все началось}
\begin{frame}{С чего все началось}
\begin{itemize}
    \item Go (часто также golang) — компилируемый многопоточный язык программирования, разработанный внутри компании Google. Был представлен в ноябре 2009 (с 2007 разработка Робертом Гризмером, Робом Пайком, Кеном Томпсоном)
    \item На данный момент поддержка официального компилятора, разрабатываемого создателями языка, осуществляется для операционных систем FreeBSD, OpenBSD, Linux, macOS, Windows, DragonFly BSD, Plan 9, Solaris, Android, AIX. Также Go поддерживается набором компиляторов gcc, существует несколько независимых реализаций.
\end{itemize}
\end{frame}
\subsection{Проблемы/Требования}
\begin{frame}{Проблемы/Требования}
%\small {Основные проблемы и требования к языку которые Роб Пайк рассматривал}
    \begin{columns}
    \footnotesize
    \column{0.5\textwidth}
    \begin{alertblock}{Проблемы}
    \centering
    {
    \begin{itemize}
        \item Медленную сборку программ
        \item Неконтролируемые зависимости
        \item Использование разными программистами разных подмножеств языка
        \item Затруднения с пониманием программ % вызванные неудобочитаемостью кода, плохим документированием и так далее
        \item Дублирование разработок
        \item Высокую стоимость обновлений
        \item Несинхронные обновления при дублировании кода
        \item Сложность разработки инструментария
        \item Проблемы межъязыкового взаимодействия
    \end{itemize}
    }
    \end{alertblock}
    \column{0.5\textwidth}
    \begin{exampleblock}{Требования}
    \centering
    {
    \begin{itemize}
        \item Ортогональность.% Язык должен предоставлять небольшое число средств, не повторяющих функциональность друг друга.
        \item Простая и регулярная грамматика.% Минимум ключевых слов, простая, легко разбираемая грамматическая структура, легко читаемый код.
        \item Простая работа с типами.% Типизация должна обеспечивать безопасность, но не превращаться в бюрократию, лишь увеличивающую код. Отказ от иерархии типов, но с сохранением объектно-ориентированных возможностей.
        \item Отсутствие неявных преобразований.
        \item Сборка мусора.
        \item Встроенные средства распараллеливания,% простые и эффективные.
        \item Поддержка строк, ассоциативных массивов и коммуникационных каналов.
        \item Чёткое разделение интерфейса и реализации.
        \item Эффективная система пакетов с явным указанием зависимостей,% обеспечивающая быструю сборку.
    \end{itemize}
    }
    \end{exampleblock}
    \end{columns}
\end{frame}
\subsection{Что получилось}
\begin{frame}{Что получилось}
    Основные возможности языка Go:
    \begin{itemize}
        \item Go — язык со строгой статической типизацией. Доступен автоматический вывод типов, для пользовательских типов — «утиная типизация».
        \item Полноценная поддержка указателей, но без возможности применять к ним арифметические операции, в отличие от C/C++/D.
        \item Строковый тип со встроенной поддержкой Unicode.
        \item Использование динамических массивов, хеш-таблиц и срезов (словарей и слайсов как в python), вариант цикла для обхода коллекции.
        \item Средства функционального программирования: неименованные функции, замыкания, передача функций в параметрах и возврат функциональных значений.
        \item Автоматическое управление памятью со сборщиком мусора.
    \end{itemize}
\end{frame}
\begin{frame}{Что получилось}
\begin{itemize}
    \item Средства объектно-ориентированного программирования ограничиваются интерфейсами. Полиморфное поведение обеспечивается реализацией интерфейсов типами. Наследование реализации отсутствует, но типы-структуры могут включать другие типы-структуры в себя.
    \item Средства параллельного программирования: встроенные в язык потоки (go routines), взаимодействие потоков через каналы и другие средства организации многопоточных программ.
    \item Достаточно лаконичный и простой синтаксис, основанный на Си, но существенно доработанный, с большим количеством синтаксического сахара.
\end{itemize}
\end{frame}